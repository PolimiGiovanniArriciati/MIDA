\lesson{Lesson 01: Stochastic Processes}{Stochastic Processes}

\begin{definition}{Stochastic Process}
	A stochastic process is a sequence of random variables all defined on the same probability space.
\end{definition}

\begin{example}
	... v(1,S), v(2,S), v(3,S), ... v(t, S), ...\\
	where: S is the outcome of a random experiment realization
					t is the time, index of the sequence
\end{example}

\begin{paragraph}
- REMARK: A SP extends the notion of a random variable to a sequence of random variables.\\
example: a deterministic time signal\\
example: a random variable in a given time
%insert awesom pictures here
\end{paragraph}

\begin{paragraph}{Widesense characterization of a SP}\\
	\begin{theorem}{mean value}
		Let $m(t)$ be the mean value of $v(t,S)$ for all $t$ and all $S$.\\
		Then $m(t)$ is a deterministic time signal.\\
		$m(t) = \mathbb{E}[v(t,S)]\\ = \int_{\Omega} v(t,S) \, dP(S)$\\
		where $P(S)$ is the probability measure of the random experiment.
		and $\Omega$ is the sample space of the random experiment.
	\end{theorem}

	\begin{theorem}{variance}
		Let $v(t,S)$ be a SP.\\
		Then $v(t,S)$ is a SP if and only if $v(t,S)$ is a SP for all $t$ and all $S$.\\
		$v(t,S) = \mathbb{E}[(v(t,S) - m(t))^2]\\
		= \int_{\Omega} (v(t,S) - m(t))^2 \, dP(S)$ or $v(t,S) = \mathbb{E}[v(t,S)^2] - m(t)^2$
	\end{theorem}
	It's the correlation between $v(t,S)$ and $v(t',S)$
	\begin{corollary}
		Remember that $gamma(t_1, t_2) = gamma(t_3, t_4i) \leftrightarrow  $t_1 - t_2 = t_3 - t_4$.
	\end{corollary}
\end{paragraph}

\newpage
